\chapter{Bayesian inference via the Ising Model}

For rare species, i.e. those with abundances below the sampling threshold, we cannot hope to use a model like the Lokta Volterra, simply because we do not have the necessary data. However, from how little we know, we cannot exclude that a species, however low in abundance, can excert a big influence on other species in the ecosystem and prove, indeed, of key global importance.
This is why investigation of the interactions among the pletora of rare species in the ecosystem is equally important.

The supplementary issue arising in this case is that the quality of our data is much worse. In fact, unless improvements in the experimental accuracy, the data we can hope for is only binary: presence/absence. 

An inferential approach of the kind showed in \ref{ch:2} cannot be applied. Here, instead, we try to employ a framework based upon Bayesian Inference that was developed recently \cite{peixoto_MDL}. The key assumption of this scheme is that the time series of binary data is well described by assuming an Ising model. However, this is only the simplest option. Indeed, the framework developed by the author is much more general and could be extended in future work to other cases as well.

Will the interactions obtained with this kind of data be coherent with those obtained through the LV? Most probably not, but for now we just make preliminary tentatives.

